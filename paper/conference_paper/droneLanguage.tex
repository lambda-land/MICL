\documentclass{sig-alternate-05-2015}
\usepackage[T1]{fontenc}
\usepackage[utf8]{inputenc}

\usepackage{cite}
\usepackage[override]{cmtt}
\usepackage[inline]{enumitem}
\usepackage[draft]{hyperref}
\usepackage{microtype}

\usepackage{lambda}
\renewcommand{\progfontsize}{\normalsize}

\usepackage{listings}

\lstset{
  language=Haskell,
  showstringspaces=false,
  basicstyle=\ttfamily\scriptsize %\bfseries
}

\begin{document}

\setcopyright{acmcopyright}


% \doi{10.475/123_4}
% \isbn{123-4567-24-567/08/06}

\conferenceinfo{IFL '16}{Aug 31--Sept 2, 2016, KU Leuven, Belgium}
% \acmPrice{\$15.00}
% \conferenceinfo{WOODSTOCK}{'97 El Paso, Texas USA}

\title{A DSL for Mixed-Initiative Drone Control}
\subtitle{[Extended Abstract]}
% \titlenote{A full version of this paper is available as
% \textit{Author's Guide to Preparing ACM SIG Proceedings Using
% \LaTeX$2_\epsilon$\ and BibTeX} at
% \texttt{www.acm.org/eaddress.htm}}}

\numberofauthors{2}
\author{
% 1st. author
\alignauthor
Keeley Abbott\\
\affaddr{School of EECS
\\ Oregon State University \\ Corvallis, OR, USA}\\
\email{abbottk@oregonstate.edu}
% 2nd. author
\alignauthor
Eric Walkingshaw\\
\affaddr{School of EECS
\\ Oregon State University \\ Corvallis, OR, USA}\\
\email{walkiner@oregonstate.edu}
}

\maketitle

\begin{abstract}
%
We present a domain-specific language embedded in Haskell for writing
mixed-initiative programs to pilot semi-autono\-mous drones. The main novelty of
our language is that it supports a flexible exchange of control flow between a
pre-programmed auto-pilot and a human pilot with manual flight controls. From
the DSL programmer's perspective, this exchange is managed by a small set of
primitives for (synchronously or asynchronously) negotiating initiative. From
the human pilot's perspective, it is supported by an architecture that
organizes the program into understandable units organized in a simple control
flow with contingencies. The design of the high-level architecture is based on
a formative study of lab protocols, which are effectively programs that must be
understood and executed by other people.
%
Our language runtime is implemented using functional reactive programming.
%
\end{abstract}


%
% The code below should be generated by the tool at
% http://dl.acm.org/ccs.cfm
% Please copy and paste the code instead of the example below.
%
\begin{CCSXML}
<ccs2012>
<concept>
<concept_id>10003120.10003121.10003124.10011751</concept_id>
<concept_desc>Human-centered computing~Collaborative interaction</concept_desc>
<concept_significance>500</concept_significance>
</concept>
<concept>
<concept_id>10003120.10003121.10003122.10003332</concept_id>
<concept_desc>Human-centered computing~User models</concept_desc>
<concept_significance>300</concept_significance>
</concept>
<concept>
<concept_id>10003120.10003123.10010860.10010911</concept_id>
<concept_desc>Human-centered computing~Participatory design</concept_desc>
<concept_significance>300</concept_significance>
</concept>
<concept>
<concept_id>10011007.10011006.10011008.10011009.10011012</concept_id>
<concept_desc>Software and its engineering~Functional languages</concept_desc>
<concept_significance>500</concept_significance>
</concept>
<concept>
<concept_id>10011007.10011006.10011050.10011054</concept_id>
<concept_desc>Software and its engineering~Command and control languages</concept_desc>
<concept_significance>500</concept_significance>
</concept>
<concept>
<concept_id>10011007.10010940.10010971.10011679</concept_id>
<concept_desc>Software and its engineering~Real-time systems software</concept_desc>
<concept_significance>300</concept_significance>
</concept>
</ccs2012>
\end{CCSXML}

\ccsdesc[500]{Human-centered computing~Collaborative interaction}
\ccsdesc[300]{Human-centered computing~User models}
\ccsdesc[300]{Human-centered computing~Participatory design}
\ccsdesc[500]{Software and its engineering~Functional languages}
\ccsdesc[500]{Software and its engineering~Command and control languages}
\ccsdesc[300]{Software and its engineering~Real-time systems software}

\printccsdesc

\keywords{}

\section{Introduction}
\label{sec:intro}

The consumer and industrial market for unmanned aircraft systems (drones) has
exploded in recent years, with the Federal Aviation Administration estimating
that 2.5 million drones will be sold in 2016 in the United States alone
\cite{FAA2016}. These vehicles have a wide range of commercial, industrial, and
scientific applications.

Drones are also an ideal platform for exploring \emph{mixed-initiative
computing}~\cite{Horvitz1999}---a model of computation in which human agents
and software agents negotiate control of a process and collaborate to solve a
problem.
%
Current consumer drone software is already mixed-initiative (usually called
semi-autonomous, in this context) since aspects of a single flight might be
controlled remotely by a human pilot and others by an on-board computer. For
example, a pilot might use a remote control to move a drone around and take
pictures, while the on-board computer makes stability adjustments, implements a
stationary hover, and automatically returns to its starting point if it loses
contact with the pilot.

Unfortunately, the majority of programs written for drones are written in
low-level languages like \prog{C}. This makes it difficult to express more
sophisticated interactions between the human pilot and the on-board computer,
and also limits potential drone programmers to experts in embedded systems.

In this paper, we present a domain-specific language embedded in Haskell for
writing mixed-initiative flight control programs for drones. The practical goal
of this language is to raise the level of abstraction for drone programming,
increasing its accessibility and making feasible new kinds of drone programs
that better exploit the strengths of both the human and computer actors in this
mixed-initiative setting.

We are also interested in mixed-initiative computing in general. Previously, we
conducted a formative study of how people write programs (in the form of lab
protocols) for other people to execute. From this study, we extracted design
insights to best support the human agent in a mixed-initiative
setting~\cite{abbott2015prog}.
%
Therefore, a higher-level goal of this work is therefore to put these design
insights to practice, evaluating whether they are actionable and lead to a
useful language design.

Our DSL is implemented using functional reactive programming (FRP), a
functional programming model based on time-varying behaviors and
events~\cite{EH97fra,WH00frp}.

The specific contributions of this work are therefore both: (1) the \emph{DSL
itself}, a high-level declarative language for a rapidly emerging application
domain, and (2) a new \emph{programming model} for mixed-initiative computing
that supports sophisticated negotiation of control between human and computer
agents.


\section{Mixed-Initiative Computing}
\label{sec:mic}

The fundamental promise of mixed-initiative computing is that humans and
computers have complementary strengths and weaknesses, and so together can
solve problems more effectively than either could alone.
%
Within the application domain of semi-automated drones (and also more
generally), computers excel at executing repetitive strategies, performing
rapid calculations, monitoring many sensors at once, and executing precise
adjustments. Meanwhile, humans excel at visual recognition, recovering from
errors, providing high-level oversight, and revising plans in in the presence
of unforeseen events.

The goal of our DSL is to allow programmers to take advantage of these
complementary strengths. For example, in fully autonomous situations, planning
for contingencies is a tedious, error-prone, and often incomplete process. A
mixed-initiative program can instead simply transfer control to the human in
such an event, relying on their judgment and creativity to recover.
%
In our previous work on lab-protocols, we found that when people write complex
instructions (programs) for other people to execute, they often provide only
high-level checks to ensure that the person executing the program is still on
the right track, and leave it up to them to recover,
otherwise~\cite{abbott2015prog}.

At a higher level, consider a program for either surveying territory or
conducting a search-and-rescue operations, which are real-world tasks that
employ drones and exemplify how the strengths of human and computer actors can
complement each other in a mixed-initiative program. Such a program might have
the drone automatically traverse an area while the human monitors camera output
for potential objects of interest, taking over manual control when one is
found. If, upon closer investigation, the object is not of interest, the human
can return control to the auto-pilot to resume traversing the area. The
computer can monitor other sensors (such as proximity and battery), and notify
the human of significant events. The human can help the drone react to
unforeseen circumstances and also adjust the high-level parameters of the
search (e.g.\ expanding or shrinking it) to incorporate new external
information, as needed.

In our study of lab protocols~\cite{abbott2015prog}, we have identified several
unique aspects of programs written for people compared to programs written for
computers, and have used related work on how humans \emph{use} plans and
protocols~\cite{Suchman1987,Lynch2002} to argue that these differences make
sense given the respective strengths and weaknesses of humans and computers.
%
We believe that a mixed-initiative programming language will be most useful if
it takes both kinds of agents into account.
%
Some distinguishing features of programs for people is that they achieve
flexibility, reusability, and robustness through \emph{simplicity}. The idea is
that, to execute a plan and adapt it to new situations, a person should be able
to understand how it works in its entirety. From this perspective, explicit
error handling actually detracts from the utility of a program since it makes
it more difficult to understand; it is better to rely on human judgment for
robustness. Similarly, programs for people are mostly linear with minimal
branching, looping, or other complex control flow.
%
However, human protocols support a rather sophisticated form of composition by
using common sense to interleave multiple protocols. This can be used in many
scenarios where a computer program would use conditionals, for example, to
interleave certain safety measures only when handling certain kinds of
dangerous materials.

% In this paper we introduce an embedded domain-specific language that is
% functionally reactive and allows for the interleaving of human actor
% directives, with those supplied by the programmer. It also allows for timed or
% event based interventions or triggers for further program execution. This
% provides a degree of variability within drone programming that takes into
% account the most up-to-date information and directives as supplied by the
% human actor, while allowing for automation of the process whenever outside
% intervention is unnecessary.

% Human actors have been used to identify visual patterns as well as in aerial
% search activities in the form of crowd-sourcing applications
% \cite{quinn2011hc}. Often these tasks come in the form of games, or provide
% some monetary compensation for the user's time in classifying or searching
% images for data. Frameworks for distributing these tasks to the crowd have
% also been developed to assist programmers in developing their queries and
% honing them through successive collections of data obtained from the crowd
% \cite{little2010turkit}.

% One difficulty that is often encountered when using crowd-sourcing for the
% purposes of surveying or search and rescue, is that the queries often take a
% long time to return results. Which can degrade the chances of a successful
% outcome, and can result in the use of outdated information when reacting to
% data collected from participants. To some extent this can be mitigated with
% the use of increased monetary incentives to the crowd participants. In
% addition to timing issues, crowd-sourcing applications often treat humans as
% monetized data processing labor, and fail to take advantage of the crowd's
% facilities beyond visual pattern matching.

% In MICL, we allow the human actor to directly interact with and manipulate the
% flow of control and the expected outcomes in real time. This provides the
% benefit of making new information gathered by the human actor's visual pattern
% matching skills more readily available in addition to allowing us to take
% advantage of the human actor's skills and knowledge that may supersede that of
% the original goals or programming.


% \subsection{Contributions}
% \label{sec:intro:contr}
% The contributions of this work are as follows:
% 
% In many cases, once a program is generated for a semi-autonomous vehicle or
% device, there is little question of what is \emph{done} with that program once
% it has been written by the programmer. Our approach allows the human actor to
% provide additional insight based on previous experience or current
% circumstances that may not have been considered by the programmer.
% 
% Our approach in MICL is to provide a set of descriptive, strongly typed and
% composable language features. The features are such that the type of the
% resulting execution is evident in the provided features. We provide examples
% of executing the language and show the simulation of moves performed by the
% system, as well as those performed by the human actor.
% 
% MICL is an \emph{embedded domain-specific language} within Haskell that
% presents a way for programmers to design mixed-initiative controller programs
% that are strongly typed. Type information for these programs is preserved,
% making it easy to ``stitch'' together existing processes into a larger flow
% for programs, as well as modify the order of events without harming the
% integrity of the overall system-provided goals.


\section{Language Overview}
\label{sec:lang}

In this section we provide a brief overview of unique aspects of the DSL (a
full description will be provided in the pre-proceedings submission).

The main design goal of our DSL is to support a flexible exchange of control
between the computer and human agents in our mixed-initiative setting. At a
low-level this is supported through four basic \emph{primitives for negotiating
control}. Although both agents are always acting in parallel, one can imagine
``control'' as token that is passed between the two agents; whoever has the
token has exclusive access to certain functionality, such as moving the drone.
%
\begin{itemize}
%
\item \textbf{Give}: Executed by the agent with the token to transfer it to the
other agent. When the computer gives control to the human, it should provide a
reason for the transfer and any information needed for the human to decide and
act. When the human gives control to the computer, it should specify where in
the program to resume executing.
%
\item \textbf{Offer}: Executed by the agent with the token to offer a transfer
of control to the other agent, which the second agent may refuse.
%
\item \textbf{Request}: Dual to offer; executed by the agent without the token
to request it from the currently in-control agent, which the in-control agent
may refuse.
%
\item \textbf{Take}: Dual to give; executed by the agent without the token to
immediately take it from the in-control agent.
%
\end{itemize}
%
To minimize the overhead of control negotiation for the human (e.g.\ minimize
annoying dialogs), offers and requests are asynchronous. That is, an offer of
control from agent A to agent B will succeed if B has made a request since A's
last offer, otherwise the runtime system assumes the offer is declined.
%
A synchronous offer can be made by giving control and then optionally giving
control back.



\iffalse
\section{Example}
\label{sec:example}
For example, a programmer codes a drone to use a flight pattern to perform
a search within a prescribed area or boundary, and in addition provides the
human actor with an intended search target and/or some set of goal-directed
tasks that need to be accomplished. We know from previous research
\cite{abbott2015prog} that users adopt human instructions or programs more
readily when they are supplied in a linear fashion (giving the user a
high-level overview of the process). These instruction sets also need to
provide users with derivation points that allow them the freedom to modify
instructions as needed, while simultaneously providing guideposts for error
recovery and locations to return to the original protocol.

\begin{lstlisting}[breaklines=true]
search :: Program
search = (takeOff (grid = (north (100 m),
                           east (100 m),
                           down (50 m)))
          (pattern = radial `fromCenter` (2 m)
           (instructions
            { waypoints = [home = center],
              goals = ["tag squirrels",
                       "suggest new search quadrant",
                       "return drone to home"
                      ]})))
\end{lstlisting}

In this example, the programmer is able to use the predefined \prog{takeOff}
program that (when supplied with a ceiling -- or \prog{down} parameter) causes
the drone to lift off from the ground at the current spot, and accelerate
upwards until it reaches the ceiling value. The \prog{grid} constructor
supplies \prog{takeOff} with the necessary ceiling parameter, but also
outlines a search grid for the drone (essentially a 100 meter by 100 meter
square).

\prog{pattern} tells the drone how to cover the prescribed grid, and
\prog{radial (2 m)} tells the drone to radiate from a given point circling
2 meters from the previous path with each pass. \prog{fromCenter} is used in
an ``inline'' fashion for convenience sake, and instructs the drone to begin
the indicated pattern from the center of the prescribed grid. Which means that
once the drone takes off, it will fly to the center of the grid, and works
it's way to the edges using the radial pattern described.

\prog{instructions} provide some tasks for the human actor to complete, and
are goal-directed in nature. These tasks provide the user some end goal they
need to reach (as well as some potential guidance if the user encounters
trouble or diverges from the initial path), but does not specify for the user
how to complete that goal. Some informational text is provided to the user via
the user interface at the start of the search, and there are guideposts for
the user to follow to ensure they are on the correct path as well.

\begin{lstlisting}
searchText :: Interface
searchText = (information
              [("tag squirrels",
                "create a new waypoint"),
               ("suggest new search quadrant",
                "use waypoint to re-center quadrant,
                 or specify new home location"),
               ("return drone home",
                "manually fly the drone back to the
                 home location, or specify new home
                 location with waypoint")
               ])
\end{lstlisting}
\fi


\section{Related Work}
\label{sec:related}
There exist languages and libraries within Haskell that deal with compiling or
writing \prog{safe-C} code, such as Ivory \cite{elliot2015ivory}, and Copilot
\cite{pike2010copilot}. Both of these language extensions are designed to be
used with controllers, and compile \prog{safe-C} code from Haskell that can be
run directly on current controller hardware.

Ivory has been used to implement code that can be run directly on Ardupilot
hardware, and has been used to generate over 200,000 lines of code for
SMACCMPilot as well. In addition, this language has been used in Boeing's
STANAG486 program to pilot an unmanned helicopter through real-world
situations \cite{boeing2016auto}. Unfortunately the structure of the language
presented by Ivory can be a bit difficult to decipher at times, and it is
unknown if the design decision to map the language closely with the compiled
code represents an additional difficulty for programmers.

Copilot on the other hand is a runtime system that generates streams of small
constant-space and constant-time \prog{C} programs that implement embedded
monitoring. Because Copilot also generates its own scheduler, there isn't any
need to implement a real-time operating system in conjunction with it.

Where both of these approaches differ from MICL, is in the availability and
representation of interactions between the system and the human actor
executing the program. In MICL we treat these interactions as first-class
citizens, rather than as an afterthought.

\TODO{Short section on where we plan to go with the paper before Aug 22nd. Can
be a small section written on Monday before we submit. Prototype of the DSL
working by the 22nd -- and in the near future actually working on a drone.}


\bibliographystyle{abbrv}
\bibliography{droneLanguage,protocolStudy}


\end{document}
