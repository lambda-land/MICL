\documentclass[11pt]{article}
\usepackage[T1]{fontenc}
\usepackage[utf8]{inputenc}

\let\labelindent\relax

\usepackage[override]{cmtt}
\usepackage{color}

\usepackage{mathptmx,amssymb,amsmath}
\usepackage{fullpage}
\usepackage[inline]{enumitem}
\usepackage{fancyvrb}
\usepackage[draft]{hyperref}
\usepackage{cite}

\usepackage{lambda}
\renewcommand{\progfontsize}{\normalsize}

\usepackage{textgreek}
\usepackage{fixltx2e}


%%%
%%% Formatting details
%%%
\sloppy
\sloppypar
\widowpenalty=0
\clubpenalty=0
\displaywidowpenalty=0
\raggedbottom
\pagestyle{plain}

% \topskip0pt
% \parskip0pt
% \partopsep0pt

\DefineVerbatimEnvironment{program}{Verbatim}
  {baselinestretch=1.0,xleftmargin=5mm,fontsize=\small,samepage=true}

\def\denseitems{
    \itemsep1pt plus1pt minus1pt
    \parsep0pt plus0pt
    \parskip0pt\topsep0pt}


%%%
%%% A few macros
%%%
\newcommand{\RevisionNeeded}{\bigskip\noindent\fbox{
\textbf{Be prepared to revise this section \emph{many} times!}}}

% \newcommand{\prog}[1]{{\small\texttt{#1}}}
% \newcommand{\bs}{\texttt{\symbol{92}}}



\begin{document}

\title{\textbf{MICL: a Mixed-Initiative Control Language}
\\ DSL Report Card}

\author{Keeley Abbott \\
School of EECS \\
Oregon State University
}

\maketitle

\section{Introduction}
\label{sec:intro}
The number of semi-autonomous controlled vehicles and devices entering the
consumer and industrial markets has increased sharply in recent years. They
have become increasingly important research tools in many fields, including
robotics, computer science, and biology in addition to military
applications. The majority of programs written to control these devices and
vehicles are written in low-level languages like \prog{C}, which limits the
amount of abstraction available to the programmers.

Here we present a domain-specific language embedded in Haskell, intended to
lift the level of abstraction available to programmers who intend to not only
utilize the abilities of the control system, but also incorporate human
input. Unlike previous languages, whose primary intent is to describe a
program executed strictly by the system, our language allows for more complex
interplay between the initiatives represented by the system, and the goals
of the users.

The specification of a program written in MICL is a description of the
tasks to be completed by the system, as well as those that require human
intervention or input (as they are seen by the programmer). In addition
MICL provides the human actor the opportunity to interleave or
intersperse their own goals and objectives as well.

\subsection{Contributions}
\label{sec:intro:contributions}
The contributions of this paper are as follows:

\begin{enumerate}[label=C\arabic*.,ref=C\arabic*,leftmargin=*]
\item a \emph{description language} for controlling semi-autonomous
  vehicles and devices that incorporates mundane instruction sets to be
  executed by the system, as well as \emph{goal-directed} instruction sets to
  be executed by the user;
\item an \emph{execution mechanism} for control sequences that allows the
  negotiation of \emph{flow control} within the defined program between the
  system and the user.
\end{enumerate}

In many cases, once a program is generated for a semi-autonomous vehicle or
device, there is little question of what is \emph{done} with that program once
it has been written by the programmer. Our approach allows the human actor to
provide additional insight based on previous experience or current
circumstances that may not have been considered by the programmer.

Our approach in MICL is to provide a set of descriptive, strongly typed and
composable language features. The features are such that the type of the
resulting execution is evident in the provided features. We provide examples
of executing the language and show the simulation of moves performed by the
system, as well as those performed by the human actor.

MICL is an \emph{embedded domain-specific language} within Haskell that
presents a way for programmers to design mixed-initiative controller programs
that are strongly typed. Type information for these programs is preserved,
making it easy to ``stitch'' together existing processes into a larger flow
for programs, as well as modify the order of events without harming the
integrity of the overall system-provided goals.


\section{Users}
\label{sec:users}
Our language is directed at programmers of automated vehicle and device
control sequences who are seeking to interject human interaction into their
routines. As a side effect, the human-readable portion of these control
sequences is heavily influenced by previous work done on how people understand
programs written by other people. \cite{abbott2015prog}

Under normal circumstances these users have a plethora of domain knowledge in
regard to writing programs for the system components they are targeting, but
may not have as much knowledge about dealing with the users of these
systems. However, they often have a good understanding of the parameters and
requirements of the programs they are writing -- where more flexibility can be
provided to the users, and where they may need additional direction in
accomplishing an assigned goal.

Users of the programs written may have little to no coding experience,
however they should not need additional technical background in order to
execute the scripts provided to them by the program. They will however need to
have some specific knowledge regarding the vehicle or device they are
assisting the operation of -- how to take manual control when necessary, and
how to return to a state of semi-autonomous control.

We intend our DSL to be a description of interactions between a programmer
(represented by a compiled program), and a human actor. As such much of what
we have done with this language is contained within simulations, and so the
code generated bears little resemblance to the low-level code that would be
necessary to actually execute these programs on controller boards. That being
said, there has already been some work done in the area of providing type safe
low-level code within Haskell. \cite{elliot2015ivory}


\section{Outcomes}
\label{sec:outcomes}
The output of programs written in our language are sets of control sequences
interpreted by the system being controlled, as well as a set of annotated
goal-directed tasks, providing the user with a protocol for the execution of
the program. At the initialization of the program, the system will begin
sending command sequences to the controller, and the user will begin executing
their tasks using the instruction set provided.

The program provides the user suggestions about where they can deviate or
alter the execution of the control sequence to accomplish any number of the
goal-directed tasks provided. In addition it provides explicit locations where
the user can or should return to a state where the control sequence can once
again return to a state of autonomous action.


\section{Use Cases / Scenarios}
\label{sec:examples}
In this section we describe the constructs used to represent semi-autonomous
device programs. We will introduce basic program constructs in
Section~\ref{sec:examples:basic} and discuss the special case of human actor
interaction in Section~\ref{sec:examples:interactive}.

\subsection{Basic Program Construction}
\label{sec:examples:basic}
At its core, MICL provides a set of data types that represent actions that can
be performed by semi-autonomous vehicles, which when strung together represent
a flight path or driving directions for the vehicle or device. If these
programs are interpreted they result in the vehicle or device servos being
actuated in such a way that they accomplish the intermediary and longterm
goals of the program. For example, two very simple programs might ask a drone
to take off and climb to a determined height, or land safely.

\begin{program}
takeOff :: State Status ()
takeOff f = do ascend fullPower `to` meters f
\end{program}

\begin{program}
land :: State Status ()
land = do descend fullPower `to` meters 0
\end{program}

We take advantage of the nature of the \prog{State Monad} to construct more
complex programs from combinations of simpler ones using Haskell's \prog{do}
notation. For example, a ``flight'' program that uses the \prog{takeOff} and
\prog{land} to safely both take off and land a drone.

\begin{program}
takeOffAndLand :: State Status ()
takeOffAndLand = do takeOff (meters 250)
                    land
\end{program}

\subsection{Interactive Program Construction}
\label{sec:examples:interactive}
In order to achieve sharing of process flow control between the program as
described by the programmer, and the goals presented to and derived by the
human actor, it is necessary to model interactions between the two in order to
negotiate this control. We represent these interactions to the user through
the \prog{Display} field in the \prog{Status}, and they are modeled as
either an instruction for the user to complete, or a waypoint for the user or
the program to arrive at. The sharing of control is represented by the
\prog{OpMode} field in the \prog{Status}, which indicates which \prog{Agent}
has control and the current \prog{Mode} of operation.

\begin{program}
type Status = (Location,Display,OpMode)

type Display = [Task]
type Task = Either Waypoint Instruction
type Waypoint = Location
type Instruction = String
\end{program}

When a task is either completed or is no longer deemed required by either
party, it can be removed from the display. Additionally, waypoints are removed
once the user or the program has visited the location associated with that
waypoint.

We provide interaction capabilities to the user through the use of the
\prog{interaction} construct, which is nested inside of another set or sets of
instructions written by the programmer. This allows the user to control the
drone, until some constraint is meant, at which point control returns to the
system and the remainder of the program is executed as needed.

\begin{program}
interactiveFlight :: State Status ()
interactiveFlight = do takeOff (meters 250)
                       forward fullPower `to` meters 250
                       strafeR fullPower `to` meters 250

                       newTask (waypoint (north 0,east 0,down 0))
                       newTask (instruction "return the drone to its home waypoint.")

                       updateOpMode (wait Human)

                       while (untilW [backward fullPower,strafeL fullPower]
                              (north 0,east 0,down 250))

                         untilO interaction (Computer,Wait)

                       land
\end{program}

From the \prog{interaction} command the user can provide additional input
signals to the drone in order to control its flight path, or to make changes
to the \prog{Display} being presented to them. These input signals are
accumulated by the state monad, and the changes are reflected to the user, and
in the continuation of the program once the user returns control to the
system.

\section{Basic Objects}
\label{sec:objects}
The basic objects of MICL include signals that are passed between the program
and the device, or between the human actor and the device. These
\prog{Signal}s represent the operating modes of the device (whether the human
or computer actor is supplying the input and whether the device or vehicle is
in a ``recovery'', ``normal'', ``return'' or ``wait'' mode), as well as the
orientation signals that control the actuation of servos on the device.

\begin{program}
data Input = Input { channel :: Agent
                   , mode :: Mode
                   , enable :: Bool
                   , roll :: Float
                   , pitch :: Float
                   , gaz :: Float
                   , yaw :: Float
                   }

type Output = Task
\end{program}

\begin{program}
type Task = Either Waypoint Instruction

type Waypoint = Location

type Instruction = String

data Agent = Human
           | Computer

data Mode = Recovery
          | Normal
          | Return
          | Wait

type OpMode = (Agent,Mode)
\end{program}

We use relative location for the device in order to determine when
waypoints have been reached in addition to assigning new waypoints. This
method of calculating the current location of the drone is known as
dead-reckoning \cite{wikipedia2016dr}.

\begin{program}
type Location = (North,East,Down)

type North = Float
type East = Float
type Down = Float
\end{program}

While we have addressed some of the temporal aspects of drone operation
through our DSL in Section~\ref{sec:comb}, another way of addressing this in
the future would be to include \emph{functional reactive programming (FRP)}
\cite{wikipedia2016frp}. In addition to providing more concrete
representations of time as well as functions for dealing with time-varying
values, FRP would provide a more concrete mechanism for reacting to user
input.

Additionally, in many cases current drone manufacturers choose to eliminate
the variable of drone orientation from their representations entirely (in fact
this is fairly standard). While this eliminates a whole host of issues when it
comes to calculating location (i.e. because north is always north you don't
have to adjust when the drone spins), it is on the whole a somewhat
unsatisfying solution. As such, there is still work to do in addressing this
issue from the standpoint of MICL.

% \TODO{We are currently still addressing the presentation of \prog{Device}s from the
% user and programmer perspective, and how that will impact the way we handle
% and process signals for different devices (i.e. terrestrial vehicles don't
% have the ability to ascend or descend, so any attempt to provide \prog{gaz}
% signals to such a device would result in errors). In the future we hope to
% either have a more concise representation of devices, or to separate out
% signal types based on the device types.

% Additionally there is a temporal aspect that has not been fully fleshed out,
% but will allow for real-time interaction between the program being run, and
% the human actor observing and/or executing additional instructions. We intend
% to accomplish this by incorporating functional reactive programming, and
% relying on streams of events and behaviors to accumulate status results,
% rather than using state driven updates.

% Lastly, we are still in the process of addressing changes in orientation and
% the effect on subsequent movements in regard to location. For the time being I
% am assuming that the orientation of the device does not effect the
% dead-reckoning location of the device -- as is the case in many commercially
% available drones.}


\section{Operators and Combinators}
\label{sec:comb}
When a \prog{Signal} is received we have to transform that into some
meaningful transformation of the \prog{State}. In order to accomplish this, we
have several operators for dealing with the different types of signals we
might expect the programmer to anticipate, as well as those signal types we
have to anticipate from the human actor. Addressing movements is the chief
among these.

\begin{program}
type Program a = a -> State (Time,Status) ()

move :: Program Input
move sig = do (t,(loc,dis,opm)) <- get
              put (time (addTime tick (t,(loc,dis,opm)))
                  ,(locate sig loc,dis,opm))
              (t,(loc,dis,opm)) <- get
              put (t,(loc,clearWaypoint loc dis,opm))
\end{program}

In any move command we have to update the current location of the device using
the device's maximum rate of travel and the throttle provided by the
programmer or the human actor.

\begin{program}
locate :: Input -> Location -> Location
locate inp loc = (locateNorth ((pitch inp) * rate)
                  (locateEast ((roll inp) * rate)
                   (locateDown ((gaz inp) * rate) loc)))

locateNorth :: Float -> Location -> Location
locateNorth f (n,e,d) = (n + f,e,d)

locateEast :: Float -> Location -> Location
locateEast f (n,e,d) = (n,e + f,d)

locateDown :: Float -> Location -> Location
locateDown f (n,e,d) = (n,e,d + f)
\end{program}

In addition, we have to increment \prog{time} as movement commands are
executed, in order to accurately reflect the amount of time that has elapsed
in the execution of a movement, and to more precisely calculate the location
using dead-reckoning. To accomplish this, we store the current ``time'' in the
\prog{State}, and update it as actions that consume time are executed.

\begin{program}
addTime :: Time -> (Time,Status) -> (Time,Status)
addTime t (t',s) = (t + t',s)

time :: (Time,Status) -> Time
time (t,s) = t
\end{program}

Crafty readers may have noticed there was an additional check for the removal
of waypoints at the end of the move command. As discussed in Section
\ref{sec:examples:interactive} waypoints are removed from the display once
they are visited, which means every time we move we have to check to see if a
waypoint has been visited and clear it.

\begin{program}
clearWaypoint :: Location -> Display -> Display
clearWaypoint _   []     = []
clearWaypoint loc (d:ds) = case d of
  (Left loc')
    | loc       == loc' -> ds
    | otherwise         -> d : clearWaypoint loc ds
  _ -> d : clearWaypoint loc ds
\end{program}

In addition to movements, we are concerned with understanding and potentially
changing the process control flow between the program and the human actor. As
such, we must also update and maintain the current signal origination and the
current mode of operation which were both discussed in Section
\ref{sec:objects}.

\begin{program}
updateAgent :: Program Input
updateAgent sig = do (t,(loc,dis,opm)) <- get
                     put (t,(loc,dis,(changeAgent opm,snd opm)))

updateMode :: Program Input
updateMode sig = do (t,(loc,dis,opm)) <- get
                    put (t,(loc,dis,(opm fst,changeMode opm)))

changeAgent :: OpMode -> Agent
changeAgent opm = case (fst opm) of
  Computer -> Human
  Human    -> Computer

changeMode :: OpMode -> Mode
changeMode opm = case (snd opm) of
  Recovery -> Normal
  Normal   -> Wait
  Return   -> Wait
  Wait     -> Normal
\end{program}

MICL is temporarily limited in the amount of interaction that is provided
between the program (representing the programmer) and the human actor. There
are further plans to address this as outlined in Section~\ref{sec:objects}
through the use of functional reactive programming. Additionally, we discuss
the temporary limitation that exists in the execution of orientation changing
commands in Section~\ref{sec:objects}.

One of the fundamental limitations of MICL is it's inability to compile down
to low-level code necessary to execute on most modern controller hardware. As
discussed in Section~\ref{sec:users}, there is a plethora of work out there
for converting Haskell code into \prog{C} code (which is the standard language
among controller hardware).


\section{Interpretation and Analyses}
\label{sec:analysis}
Execution of simulations of programs within MICL is handled through the GHCi
terminal screen using the \prog{State} monad, and the \prog{runState}
function. When programs are executed, they also take advantage of Haskell's
\prog{IO} monad to display status information to the user, as well as tasks
and waypoints provided by the program. \prog{IO} is also the means of
receiving commands from the user, and transmitting the signals back to the
drone.

In order to execute a program in MICL, the user provides the program to be
run, and an initial \prog{state} for the program to operate on. In MICL we
provide a \prog{statusDefault} construct that initiates the drone with a
starting location of \prog{(north 0,east 0,down 0)}, an empty beginning
display, and an \prog{OpMode} of \prog{(Computer,Normal)}.

\begin{program}
statusDefault :: (Time,Status)
statusDefault = (seconds 0.0,((north meters 0.0,east meters 0.0,down meters 0.0)
                             ,[],(Computer,Normal)))

> runState simpleGrid statusDefault
((),(62.0,((0.0,0.0,0.0),[],(Computer,Normal))))
\end{program}

Interrupts in the program provide the user with an \prog{IO} screen with
command options for the user to execute.


\section{Cognitive Dimension Evaluation}
\label{sec:cogdim}
\subsection{Abstraction Gradient}
\label{sec:cogdim:abstr}
MICL is an \emph{abstraction-tolerant} language that permits the use of
constructs exactly as they come, however owing to its affinity for composing
functions and programs, it allows for the creation of additional
abstractions. The level of abstraction available from MICL is initially low,
which allows it to be used by less experienced programmers more easily. Novice
programmers can start by understanding the high-level concepts behind signals.

\subsection{Closeness of Mapping}
\label{sec:cogdim:map}
Problems that can be modeled in MICL as programs are mapped relatively closely
to those that can be observed in the real world. We are able to map entities
in the user's task domain directly to task-specific program entities, and
operations on the program entities can be likewise mapped directly onto
program operations.

\subsection{Consistency}
\label{sec:cogdim:cons}
One area that MICL has issues in accomplishing is consistency. While the
constructs within the language are fairly easy to understand once they have
been learned, it may be more difficult for users to ``guess'' at additional
constructs they may not have learned yet. In addition, the \prog{do} notation
that we rely on may present a barrier to users of other languages where this
type of notation has a different meaning.

\subsection{Diffuseness/Terseness}
\label{sec:cogdim:diff}
MICL has a relatively compact field of notation to achieve resulting programs
written in the language. As such, and also owing to its relative closeness of
mapping with the problem world, it offers a fairly terse representation to
users and programmers. MICL requires relatively few lexemes to achieve
resulting programs.


\section{Implementation Strategy}
\label{sec:implementation}
Using a shallow embedding allows us to take advantage of Haskell's \prog{do}
notation as a way of representing program syntax within MICL. Additionally, we
have the luxury of easily composing programs that have already been written
by using of \prog{State} management using monads. However, because MICL does
not compile to low-level code there is some additional work that would need to
be done in order for programs to be run directly on vehicle or device
hardware.

Implementation of MICL in a \emph{language workbench} might be advantageous,
because we could compose MICL with some similar EDSLs that do compile down to
low-level code that can be run directly on controller hardware.  One example
of a language that it would be advantageous to compose MICL with would be
Ivory \cite{elliot2015ivory}. The framework is already there to compile
Haskell code to \prog{safe-C}, and by taking advantage of this we need not
compile the code from directly within MICL.


\section{Find Similar/Related DSLs}
\label{sec:related}
As we mentioned in Section \ref{sec:users} there exist languages and libraries
within Haskell that deal with compiling or writing \prog{safe-C} code, such as
Ivory \cite{elliot2015ivory}, and Copilot \cite{pike2010copilot}. Both of
these language extensions are designed to be used with controllers, and
compile \prog{safe-C} code from Haskell that can be run directly on current
controller hardware.

Ivory has been used to implement code that can be run directly on Ardupilot
hardware, and has been used to generate over 200,000 lines of code for
SMACCMPilot as well. In addition, this language has been used in Boeing's
STANAG486 program to pilot an unmanned helicopter through real-world
situations \cite{boeing2016auto}. Unfortunately the structure of the language
presented by Ivory can be a bit difficult to decipher at times, and it is
unknown if the design decision to map the language closely with the compiled
code represents an additional difficulty for programmers.

Copilot on the other hand is a runtime system that generates streams of small
constant-space and constant-time \prog{C} programs that implement embedded
monitoring. Because Copilot also generates its own scheduler, there isn't any
need to implement a real-time operating system in conjunction with it.

Where both of these approaches differ from MICL, is in the availability and
representation of interactions between the system and the human actor
executing the program. In MICL we treat these interactions as first-class
citizens, rather than as an afterthought.


\section{Design Evolution}
\label{sec:evol}
When we originally began working with the \prog{State} monad, we thought of a
\prog{Program} as simply a \prog{Signal -> State Status ()}, however as we
worked with this design, it became increasingly apparent that we were going to
have to take \prog{Time} into account.

\begin{program}
type Program = Signal -> State (Time,Status) ()
\end{program}

This allowed us to include time-based functions that would execute a given
program at a specified time as well, or react to the passage of time in
different ways.

\begin{program}
at :: Program Input -> Time -> Input -> State (Time,Status) ()
at prg t inp = do (t',s) <- get
                  case (t == t') of
                    True  -> prg inp
                    False -> return ()

untilT :: [Input] -> Time -> State (Time,Status) ()
untilT []     tme = do return ()
untilT (i:is) tme = do (t,s) <- get
                       case (t == tme) of
                         False -> i `to` (projectLoc i s) >> untilT is tme
                         True  -> return ()
\end{program}

In addition, we realized there were different signal types we would have to
take into consideration -- \prog{Input} and \prog{Output}
signals. \prog{Input} signals are those that provide instructions to the drone
in order to actuate motors and control the amount of power being sent to
them. \prog{Output} signals are those that provide instructions and
information to the user about the goal-directed tasks.

\begin{program}
type Program a = a -> State (Time,Status) ()
\end{program}

One advantage of this design, is that we can accommodate other types of
\prog{Signals} provided by the programmer that may not have already been
considered in the current design.


\section{Future Work}
\label{sec:future}
In order to resolve the temporary limitations of the language mentioned in
Section~\ref{sec:comb} we intend to implement MICL as a deeply embedded FRP
language in Haskell. The long-term goal being to provide more seamless
interaction between the system and the human actor, as well as allow us to
perform additional analysis on the language.

Another advantage of incorporating FRP into the design of the language is the
ability to further refine the \prog{Display} system within MICL. Most FRP
libraries also include visualization extensions that will allow us to provide
the user with a more polished GUI. This will also give us the ability to
display the user instructions and waypoints in a more human-readable
protocol-like manner.

Currently MICL is focused on the domain of drones, but there are other
controller environments that could take advantage of the language we have
built. In order to do so, we need to generalize the language constructs
further, and this can potentially be assisted through the \emph{deepening} of
the language into a \emph{final} form.


\bibliographystyle{abbrv}
\bibliography{micl}



\end{document}
